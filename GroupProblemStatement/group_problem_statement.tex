\documentclass[onecolumn, oneside, letterpaper, draftclsnofoot, 10pt, compsoc]{IEEEtran}

\usepackage{pdfpages}
\usepackage{graphicx}

\usepackage{url}
\usepackage{setspace}

\usepackage{amssymb}
\usepackage{amsmath}
\usepackage{amsthm}
\usepackage{alltt}
\usepackage{color}
\usepackage{enumitem}
\usepackage{textcomp}

\usepackage[margin=0.75in]{geometry}

\parindent = 0.0 in
\parskip = 0.0 in

% 1. Fill in these details
\def \CapstoneTeamName{         Beaver Hawks}
\def \CapstoneTeamNumber{       14}
\def \GroupMemberOne{           Anton Synytsia}
\def \GroupMemberTwo{           Matthew Phillips}
\def \GroupMemberThree{         Shanmukh Challa}
\def \GroupMemberFour{          Nathan Tan}
\def \CapstoneProjectName{      American Helicopter Society Micro Air Vehicle Competition}
\def \CapstoneSponsorCompany{   Potentially Columbia Helicopters}
\def \CapstoneSponsorPerson{    Nancy Squires}

% 2. Uncomment the appropriate line below so that the document type works
\def \DocType{      Problem Statement
                %Requirements Document
                %Technology Review
                %Design Document
                %Progress Report
                }

\newcommand{\NameSigPair}[1]{\par
\makebox[2.75in][r]{#1} \hfil   \makebox[3.25in]{\makebox[2.25in]{\hrulefill} \hfill        \makebox[.75in]{\hrulefill}}
\par\vspace{-12pt} \textit{\tiny\noindent
\makebox[2.75in]{} \hfil        \makebox[3.25in]{\makebox[2.25in][r]{Signature} \hfill  \makebox[.75in][r]{Date}}}}
% 3. If the document is not to be signed, uncomment the RENEWcommand below
%\renewcommand{\NameSigPair}[1]{#1}

%%%%%%%%%%%%%%%%%%%%%%%%%%%%%%%%%%%%%%%
\begin{document}
\begin{titlepage}
    \pagenumbering{gobble}
    \begin{singlespace}
        \includegraphics[height=4cm]{coe_v_spot1}
        \hfill
        % 4. If you have a logo, use this includegraphics command to put it on the coversheet.
        %\includegraphics[height=4cm]{CompanyLogo}
        \par\vspace{.2in}
        \centering
        \scshape{
            \huge CS Capstone \DocType \par
            {\large\today}\par
            \vspace{.5in}
            \textbf{\Huge\CapstoneProjectName}\par
            \vfill
            {\large Prepared for}\par
            \Huge \CapstoneSponsorCompany\par
            \vspace{5pt}
            {\Large\NameSigPair{\CapstoneSponsorPerson}\par}
            {\large Prepared by }\par
            Group\CapstoneTeamNumber\par
            % 5. comment out the line below this one if you do not wish to name your team
            \CapstoneTeamName\par
            \vspace{5pt}
            {\Large
                \NameSigPair{\GroupMemberOne}\par
                \NameSigPair{\GroupMemberTwo}\par
                \NameSigPair{\GroupMemberThree}\par
                \NameSigPair{\GroupMemberFour}\par
            }
            \vspace{20pt}
        }
        \begin{abstract}
        % 6. Fill in your abstract
The American Helicopter Society Micro Air Vehicle project's objective is to build an aerodynamically efficient helicopter that can be remotely controlled to complete an obstacle course. The team is comprised of Mechanical Engineers, Electrical and Computer Engineers, and Computer Scientist. The Mechanical Engineers are responsible for building an aerodynamic body for the helicopter. The Electrical Engineers are making a breadboard to support the cameras and sensors, while also managing power consumption and firmware programming. Finally, the Computer Science majors are developing software to provide basic flight controls and a unified structure for displaying and processing sensor data. These basic features can be iterated into more complex functionality. Some ideas include an obstacle collision warning system and autonomous object pick up and drop off elements.
        \end{abstract}
    \end{singlespace}
\end{titlepage}
\newpage
\pagenumbering{arabic}
\tableofcontents
% 7. uncomment this (if applicable). Consider adding a page break.
%\listoffigures
%\listoftables
\clearpage

% 8. now you write!
\section{Competition Description and Definitions}
The competition is an obstacle course with a ten-minute time limit that will be held on May 2019 in Philadelphia, PA. It is sponsored by Lockheed Martin, Columbia Helicopters, Sikorsky, Bell, and Eurocopter. The course starts with a team\textquotesingle s vehicle picking up a package, the vehicle will then carry the package through the entirety of the course without incident. The vehicle then drops off the package at the designated drop area, without destroying the elements in the package. Then, the vehicle will return to the starting point of the course. A portion of the course will include an area where the aircraft is not visible to the pilot. The helicopter must be equipped with sophisticated cameras and sensors that use computer vision algorithms to help the pilot guide the vehicle without proper visibility. These algorithms will include Convolutional Neural Networks with convolutional and dropout layers.

\begin{description}
\item[Computer Vision] A subset of artificial intelligence that deals with providing information through visual and image data.

\item[Convolutional Neural Network (CNN)] A different kind of Neural Network that has learnable weights and biases that help to inference the data better. A CNN learns from each raw pixel from an image and assigns new weights to the convolution layer.

\item[Convolutional Layer] A layer of a CNN that performs the computations.

\item[Dropout Layer] A layer of the CNN that prevents overfitting of data.
\end{description}


\section{Proposed Solution}
During last year's competition, the helicopter blew up because the motors were overworked. Additionally, the aircraft had too many autonomous features that caused the pilot to lose control of the aerial vehicle. Due to these mistakes, our team decided to remove unnecessary autonomous elements and focus on building a light-weight aircraft with sophisticated computer vision and collision avoidance technology to help the pilot fly the helicopter with ease. Also, the team had problems navigating to the payload pickup point and effectively pick up the package. Though the primary intent of this competition is to win, the purpose of our helicopter designs is to influence more sophisticated helicopter designs in the future.

To ease pilot’s navigation through an obstacle course, we propose building intelligence for the aircraft using computer vision, range sensors, and artificial intelligence. Information includes collision avoidance and pathfinding software. Collision avoidance will either warn the pilot of collisions or overwrite controls to maintain a safe distance from an obstacle. The navigation software will suggest a route to take that avoids any obstacles with a safe length and safe height. Additionally, we will be creating an autonomous payload pickup functionality. With the use of cameras and sensors, the user will bring the helicopter to the pickup point in a position that's easiest to pull the package. Finally, the computer vision software will autonomously drop off the payload by guiding the helicopter to the ideal location to disembark the package. These features together will help the aircraft complete the obstacle course while picking up a payload and finishing at a good time.

\section{Performance Metrics}
The metrics for the helicopter include building a streamlined aircraft that is powerful enough to sustain the weight of two cameras and sensors and maintain a total aircraft weight of 500 grams. The aircraft must also carry a raspberry pi or an Arduino that contains a high-quality transmitter that provides visual data of the helicopter path with minimal latency. If the antenna has too much latency, then the pilot will receive delayed feedback and might hinder their ability to guide the helicopter through obstacles. We will use the Python and C++  programming languages to process the data being transmitted from the aircraft. To the aerial vehicle steady, our helicopter will read input from two Inertial Measurement Unit (IMU) sensors which the raspberry pi will use to help balance the chopper. The vehicle should handle collision avoidance with static, and potentially moving, obstacles.

\subsection{Hovering}
The primary objective is to have the helicopter hover in place. When no input is given from its remote, the helicopter should fly at its current location. We will use three sensors to do this: a gyroscope, GPS , and an accelerometer. Reasons for each are described below:
\begin{description}
\item[Gyroscope] A gyroscope is needed to help stabilize all three axes of alignment of the coaxial helicopter. The device will prevent the aircraft from flipping over and will also serve as the primary sensor for countering winds.
\item[Global Positioning System (GPS)] To stabilize the helicopter at a particular height, the input from the GPS sensor will allow the vehicle to maintain a persistent altitude autonomously.

\item[Accelerometer] To keep the helicopter stable, there is a need for small reading changes in the motion of the helicopter. The addition of the accelerometer data will significantly improve the helicopter’s ability to maintain a stable position.
\end{description}
With the three sensors in place, a software could be written to maintain a helicopter in one place.


\subsection{Collision Avoidance}
If there is an obstacle to the left, right, front, or rear of the helicopter, the helicopter should automatically maintain a safe distance from the barrier. To achieve obstacle collision avoidance, we suggest including three or four range sensors at the horizontal plane of the helicopter, with each facing outward, in equally distributed directions.

If there is an obstacle in the direction the helicopter is being controlled, the aircraft should automatically maintain its safe distance from the barrier and not hover into the proposed course.

\subsection{Vertical Distance Control}
There are two types of vertical distance controls: safe height and additive height.
\begin{description}
\item [Safe Height] Safe distance is also known as the minimal distance, which is used to control the helicopter's minimum height from the surface below. If helicopter height becomes smaller than the minimal height, a lift will be generated to bring the aircraft back to the safe height. Some exceptions may apply, for example, when the helicopter is encountered with an obstacle, the pilot guides the vehicle to a safe altitude.
\item [Additive Height] If there is a barrier in front of the helicopter, the operator must lift the aircraft by adding height. The height sensor does not maintain additive altitude.
\end{description}

\subsection{Autonomous Handle of Package}
In addition to navigating the helicopter, there must be a feature on the remote to acquire an identifiable package at the designated location.

\subsubsection{Necessary Components}
For picking up a package, the helicopter must have a rope and an electromagnet, or a hook attached. The payload itself should have a metal marker or a hook. For the pilot to pick up the item with ease, there must be a ground facing camera to help navigate to the ideal position.

\subsubsection{Acquirement}
When the pickup feature is activated, the image recognition software first checks if there is a package. If there is no package or the package is beyond the proximity of the helicopter's vision, then the software guide the aircraft to a better location. If there is an identifiable package, the image recognition software will aide the chopper in positioning for pick up. The height sensor will aide the helicopter in descending over the payload. Once the aircraft reaches the ideal pickup position, it will first descent to a safe height,  engage the hook, grab the payload, and return to an optimal operating altitude. If the pickup is unsuccessful, the pilot can reposition the aircraft using the software and try again.

\subsubsection{Release}
To release a package, the operator will have one button to do so. Once pressed, the helicopter will descend, turn off the electromagnet and then ascend to a safe height.

\end{document}
